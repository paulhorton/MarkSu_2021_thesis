近年來由於次世代定序技術(NGS)的出現,使得我們可以同時對大量DNA進行定序,大幅降低定序成本,許多相關的研究也跟著大幅擴展,NGS數據中變異調用幾乎是所有下游分析和解釋過程都依賴的關鍵步驟,但是對於準確基因變異體偵測仍然有許多問題及挑戰。因此我們先前提出過一個工具Explicit Alternative Genome Likelihood Evaluator(EAGLE),透過使用顯示概率模型,處理變異調用過程中固有的不確定性。明確評估測序數據與推定變異所隱含的替代基因組序列的匹配程度。但是不論是EAGLE的模型抑或是一般的變異調用方法,或多或少都仰賴將讀序映射到參考基因組序列上的結果。而映射的方法是透過將讀序與參考基因組序列比對,根據比對的相似程度決定他們可能來自參考基因組序列上的位置。

由於我們比對的目標參考基因組序列是線性的,僅包含單個序列,因此無法捕獲基因組的多樣性。所以比對會出現可能一定程度的誤差,造成我們讀序可能被映射到錯誤的位置或是無法映射,這種現象我們稱之參考偏差。錯誤的讀序映射會導致假陰性或假陽性變異調用,進一步影響到我們對於辨認基因體變異的準確性。而先前已經有許多研究討論消除參考偏差所帶來的影響,其中一種方法就是透過改進參考基因組序列,使其能夠包含基因組的多樣性,從而達到更好的映射結果。本文中我們提出一個類似概念的方法,並應用於我們先前提出的工具-EAGLE上,探討我們提出的方法對於降低參考偏差帶來的影響是否可行以及其潛力。

我們的方法可以分成五個部分,第一個部分受透過基因突變資料(VCF)中取得變異點資訊,結合參考基因組序列建立假設序列。第二個部分是為讀序檔案(FASTQ)建立一個能夠快速搜尋的索引結構。第三個部分是透過Burrows-Wheeler Aligner(BWA)以及我們上一部份建立的讀序索引結構在讀序檔案中找出所有能夠與我們建立的假設序列匹配的讀序。第四個部分則是檢查原本的映射結果檔案(BAM)中的堆積序列,堆積序列是在映射結果中被映射到相同區域的所有讀序,檢查是否存在原有的堆積序列中,對那些不存在堆積序列中的讀序建立相關的比對資訊。第五部分則是將我們找到的讀序以及比對資訊加入堆積序列中,改進我們EAGLE的計算結果。

在實驗中,我們模擬參考偏差發生的情景,透過我們的方法驗證是否能找到那些受到參考偏差的影響而遺失的讀序,結果顯示我們能夠找到當中大部分被遺失的讀序,但是在一些重複序列過多的區域可能有所限制。我們同時也在 dbSNP的變異資料及上進行測試,基本上也能夠取得相似的結果,惟在單核苷酸多態性的變異上效果沒有達到理想的成效。但大體而言,綜合我們所增加的時間以及成效,我們認為我們的方在在解決參考偏差的影響是相當有潛力的,但是應用到實際情況還是需要更進一步的驗證以及謹慎的解釋。

\begin{flushleft}
\mbox{{\bf 關鍵字}: 次世代定序、變異點偵測、參考偏差 }
\end{flushleft}
